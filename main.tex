\documentclass{aip-cp}

\usepackage[numbers]{natbib}
\usepackage{rotating}
\usepackage{graphicx}

% Document starts
\begin{document}

% Title portion
\title{Dynamics of the real estate market: determining factors in the value of housing in Portugal}

\author[aff1]{Maria J. Ferreira}
\author[aff1]{Aldina Correia\corref{cor1}}

\author[aff1]{Vanda Lima}
\eaddress{vlima@estg.ipp.pt, URL: https://ciicesi.estg.ipp.pt/}
\author[aff2]{Diogo Ribeiro}
\eaddress{diogo.ribeiro@mysense.ai, URL: https://www.mysense.ai}
\affil[aff1]{CIICESI, ESTG, Politécnico do Porto 
Rua do Curral, Casa do Curral, Margaride, 4610--156 Felgueiras -- Portugal 
}
\affil[aff2]{MySense, UK}
\corresp[cor1]{Corresponding author: aic@estg.ipp.pt}

\maketitle

\begin{abstract}
The present study aims to support decision-making at individual level and at the level of political decision-makers at the level of housing in Portugal. At the individual level, it intends aid in the purchase/investment decision, either of the buyer, as well as the investor.
At the level of political decision makers, whether they are municipal, inter-municipal or national, it intends to be an aid in decision-making on the best policies for the promotion, renewal and rehabilitation of housing in their intervention areas.
In this work data available in PORDATA -- Portugal Contemporary Database of all 308 municipalities in Portugal were used. The methodology adopted consists in a multiple linear regression, in order to understand which factors most influence the average value of the dwellings traded by municipality. Subsequently, an analysis was carried out in order to identify possible clusters within the municipalities.
Factors considered include variables related to location of municipalities, population density of the municipalities, purchasing power of the inhabitants of the municipalities, average size of families in the municipalities, unemployment in municipalities, environmental quality in municipalities, aging in municipalities, average level of education in municipalities, civil status of the inhabitants of the municipalities, own housing by the owner in the municipalities, existing housing in municipalities. 
The results suggest that only population density and environmental quality have no significant influence on the average value per municipality of transacted housing. 2, 3 or 4 clusters of municipalities can be considered, always highlighting a group of municipalities with a much higher housing price than the others.

\textbf{Keywords}: Housing Price, Portugal, Multiple Linear regression, Cluster Analysis.
\end{abstract}

% Head 1
\section{INTRODUCTION}

Housing is not only a fundamental human right but also a critical component of a stable and thriving society. The provision of safe, affordable, and stable housing forms the bedrock upon which individuals and families build their lives. The significance of housing was brought into sharp relief following the 2007 financial crisis, an event that precipitated a seismic shift in the global economic landscape. The crisis exposed vulnerabilities in the financial system, resulting in stringent credit conditions and a liquidity shortfall. These financial disturbances had a domino effect, causing widespread economic turmoil. The availability of housing finance became constrained, leading to a decline in homeownership and a spike in foreclosures. The aftershocks of this economic upheaval were felt across various sectors, underscoring the intricate link between housing markets and financial stability. As we navigate the post-crisis world, the lessons learned continue to shape policies and frameworks aimed at ensuring that the right to housing is safeguarded against such systemic shocks in the future.

The evolution of society has brought with it a heightened awareness of what constitutes essential human needs, among which housing stands out as a critical factor. As a result, there is a growing demand for housing that not only provides shelter but also meets a range of specific requirements that align with contemporary living standards and personal aspirations. These requirements often encompass considerations such as location, accessibility, environmental sustainability, and technological integration.
This evolving demand compels a closer examination of the diverse factors that drive individual and collective choices in the housing market. Factors such as demographic changes, economic conditions, cultural trends, and policy developments all play a part in shaping housing preferences and, by extension, the real estate landscape. For example, a societal shift towards remote work may increase the desire for homes with office spaces, while urbanization trends might heighten the value placed on proximity to amenities and transportation.
Understanding these factors is not only academically interesting but also practically vital for stakeholders, including policymakers, urban planners, real estate developers, and financial institutions. Insights into housing preferences can inform strategies to address housing shortages, guide the development of new housing projects, influence zoning laws, and inform mortgage lending criteria.
Moreover, these preferences have a direct correlation with the average value of housing. When certain features become highly sought after, properties that possess these attributes may see an increase in value, potentially altering the market dynamics. Conversely, houses lacking these desired characteristics may experience a decrease in perceived value, impacting affordability and availability.
In essence, the intersection of human needs and housing choices is a dynamic area of study that has far-reaching implications for individuals, communities, and the broader economy. It is a subject that requires continuous research to ensure that the housing market can adapt and respond to the changing tapestry of societal needs.

The primary goal of this study is to determine the factors that influence the value of housing in municipalities in Portugal. This is an econometric study aimed at assessing the impact of various characteristics of the municipalities on the average value of housing transactions carried out in these municipalities. It also seeks to understand how the variation of these municipal characteristics influences the evolution of prices. Additionally, the objectives of this work include studying the behavior of housing value fluctuations in Portugal and determining the similarities between municipalities regarding these fluctuations.

Existing research on the determinants of housing values has largely concentrated on broad geographic scopes, such as international comparisons, regional trends, or areas surrounding major urban centers. There has been a notable lack of in-depth analysis at finer scales like that of municipalities. This gap in the literature is what the current study seeks to address. It endeavors to carry out a detailed national-level analysis, with a focus on the individual municipalities within Portugal. The objective is to categorize these municipalities based on the Average Market Value of Property Transactions (VMPT) and to pinpoint the specific factors that exert an influence on these values. By doing so, the study aims to provide a nuanced understanding of the local variables affecting housing prices within the distinct Portuguese municipalities.

This research offers significant insights and applications that inform decision-making processes for a range of stakeholders. On the individual level, it equips buyers and investors with data-driven guidance, facilitating more informed and strategic market engagements. At the collective level, it serves as a resource for policymakers, providing an empirical foundation for the development of housing policies. Consequently, this study not only enriches the theoretical discourse in academic circles but also stands as a practical tool for real-world application in the housing sector.



\section{\uppercase{Factors influencing the value of housing}}

O que diz a revisão de literatura 

% Head 2
\subsection{\uppercase{Data and Results}}
\subsubsection{Data}

\begin{itemize}
    \item Amostra
    \item Caracterização da Amostra - Tabelas descritivas 
\end{itemize}







\subsubsection{Factors in
the value of housing in Portugal}

\begin{itemize}
    \item Segundo modelo de regressão
\end{itemize}



% Head 3
\subsubsection{Municipalities classification
}

\begin{itemize}
    \item Análise de Clusters
\end{itemize}


% Sections that will go in second font
\subsection{\uppercase{Conclusions}}

The analysis presented in this paper provides a detailed exploration of the factors influencing housing prices across municipalities in Portugal, with a distinct emphasis on the period of transition from growth to stabilization in 2017. Through a refined multiple linear regression approach, we have delineated the impact of geographical, demographic, economic, and educational variables on property values within local contexts.
Our findings reveal a pronounced geographic determinant in housing valuation, asserting the premium placed on coastal and urban locations. The size of the average household emerges as a significant predictor, reflecting the market's response to family housing demands. Economic health, as measured by local unemployment rates, exhibits a predictable yet critical inverse correlation with housing prices, indicating the sensitivity of the real estate market to job availability. Moreover, the positive association between educational levels and housing prices emphasizes the value attributed to intellectual capital in residential desirability.
Notably, the study has highlighted the nuanced effect of homeownership rates on the market, contesting the straightforward view of ownership as a mere indicator of market heat. Instead, it is shown to interact with other factors in a way that could point to market maturity or saturation, suggesting a more complex scenario than previously understood.
These results carry implications for stakeholders in the housing market. Potential homebuyers can benefit from understanding how these variables interconnect, potentially guiding more informed decision-making. For policymakers, the insights offer concrete evidence to inform strategic planning, economic development initiatives, and educational investment to foster balanced growth and maintain market vitality.
The static nature of the data from a single year is a recognized limitation of this study, suggesting the need for ongoing research. As new census data becomes available, it will be imperative to re-evaluate these findings and observe how the interplay of these factors evolves over time. This continued analysis will be vital for adapting policies to the shifting realities of Portugal's diverse municipalities and ensuring that interventions remain relevant and effective in stabilizing and stimulating the housing market.


% Acknowledgement
\section{ACKNOWLEDGMENTS}
This work has been supported by national funds through FCT - Fundação para a Ciência e Tecnologia through project UIDB/04728/2020.

% References

\nocite{*}
\bibliographystyle{aipnum-cp}%
\bibliography{sample.bib}%


\end{document}
