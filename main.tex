\documentclass{aip-cp}

\usepackage[numbers]{natbib}
\usepackage{rotating}
\usepackage{graphicx}

% Document starts
\begin{document}

% Title portion
\title{Dynamics of the real estate market: determining factors in the value of housing in Portugal}

\author[aff1]{Maria J. Ferreira}
\author[aff1]{Aldina Correia\corref{cor1}}

\author[aff1]{Vanda Lima}
\eaddress{vlima@estg.ipp.pt, URL: https://ciicesi.estg.ipp.pt/}
\author[aff2]{Diogo Ribeiro}
\eaddress{diogo.ribeiro@mysense.ai, URL: https://www.mysense.ai}
\affil[aff1]{CIICESI, ESTG, Politécnico do Porto 
Rua do Curral, Casa do Curral, Margaride, 4610--156 Felgueiras -- Portugal 
}
\affil[aff2]{MySense, UK}
\corresp[cor1]{Corresponding author: aic@estg.ipp.pt}

\maketitle

\begin{abstract}
The present study aims to support decision-making at individual level and at the level of political decision-makers at the level of housing in Portugal. At the individual level, it intends aid in the purchase/investment decision, either of the buyer, as well as the investor.
At the level of political decision makers, whether they are municipal, inter-municipal or national, it intends to be an aid in decision-making on the best policies for the promotion, renewal and rehabilitation of housing in their intervention areas.
In this work data available in PORDATA -- Portugal Contemporary Database of all 308 municipalities in Portugal were used. The methodology adopted consists in a multiple linear regression, in order to understand which factors most influence the average value of the dwellings traded by municipality. Subsequently, an analysis was carried out in order to identify possible clusters within the municipalities.
Factors considered include variables related to location of municipalities, population density of the municipalities, purchasing power of the inhabitants of the municipalities, average size of families in the municipalities, unemployment in municipalities, environmental quality in municipalities, aging in municipalities, average level of education in municipalities, civil status of the inhabitants of the municipalities, own housing by the owner in the municipalities, existing housing in municipalities. 
The results suggest that only population density and environmental quality have no significant influence on the average value per municipality of transacted housing. 2, 3 or 4 clusters of municipalities can be considered, always highlighting a group of municipalities with a much higher housing price than the others.

\textbf{Keywords}: Housing Price, Portugal, Multiple Linear regression, Cluster Analysis.
\end{abstract}

% Head 1
\section{INTRODUCTION}
Housing is crucial for personal and societal stability, a fact underscored by the 2007 financial crisis which highlighted the connections between housing markets and economic stability. The crisis led to tougher credit, reduced liquidity, and economic distress, affecting homeownership and increasing foreclosures. Post-crisis, ensuring the right to housing against such shocks has become a policy focus.

Societal changes increase the demand for housing that meets modern standards and personal needs, including location, sustainability, and technological needs. This demand shapes market preferences influenced by demographic shifts, economic trends, cultural changes, and policy reforms. Such insights are vital for stakeholders to address housing issues, with preferences affecting property values and market dynamics.

The study focuses on how municipal characteristics in Portugal affect housing values, filling a research gap on local-level analysis. It assesses the influence of these traits on price trends and categorizes municipalities by average transaction values. This research aids buyers, investors, and policymakers with data for better decision-making and informs housing policy development.

\section{\uppercase{Factors influencing the value of housing}}

This paper includes a bibliometric analysis conducted in January 2021, using the Web of Science database to explore the prevalence of "housing prices" as a research topic. The initial search yielded 2,369 records, which, after refining to only scholarly articles, resulted in 2,003 publications for review. The earliest article dates back to 1972, with a marked increase in interest observed from 1995 and a significant uptick in research activity in the early 21st century. This growing interest likely correlates with global housing market volatility and the disparity between housing prices and income growth.
As of June 30, 2021, the number of articles published in 2021 alone reached 140, representing 6.272% of all publications on this topic to date. The authors contributing most prolifically to this field include Hui, Tsai, Gupta, Song, Wen, Glaeser, Su, Zhang, Chau, and Quigley, reflecting a diverse geographic interest spanning the USA, China, Spain, South Korea, the UK, Taiwan, Australia, Canada, and Portugal.
The paper also delves into the determinants of transactional housing values, highlighting contributions from both international and regional studies. A special focus is placed on municipal-level research, which remains sparse, suggesting that this work provides both innovative and theoretical contributions. A detailed examination of structural and political factors affecting housing values from 2000 to 2015 is also provided.
In summarizing the factors affecting housing prices, location emerges as a key determinant. It's noted that quality of life, influenced by factors such as proximity to natural resources and amenities, significantly impacts housing values. Comprehensive studies using hedonic pricing models and GIS for spatial adjustments offer insights into environmental and neighborhood attributes' effects on prices. For instance, the proximity to destinations, views, and air quality have been emphasized as pivotal in determining the willingness to pay for housing. The paper underscores the relevance of location in housing acquisition, evidenced by studies on Istanbul and Ankara's housing markets, and corroborates these findings with extensive data analysis across various Spanish provinces using STAR and GLM methodologies.


Population Density and Housing Market Dynamics: A Synopsis of Influential Factors
The population dynamics of Portugal, as reported by the National Statistics Institute (INE) in 2017, reveal a gendered count of 10,291,027 with a declining trend since 2010, slightly mitigated in recent years. An aging demographic is evident with a rise in the median age from 42.7 to 44.2 years between 2012 and 2017. Zhou et al. (2020) identified a significant positive correlation between housing prices and population density in 283 Chinese cities, positing that higher housing prices and denser populations are pivotal for increased productivity.
Rappaport (2008) highlighted the influence of population density on U.S. housing prices, observing various endogenous factors that co-vary with it, such as a slight decrease in wages, moderate house price increases, and sharp land price ascensions. High quality of living seems to be capitalized in land and house prices rather than wages. Calcagno et al. (2009) found that Italian household consumption correlates with housing capital gains, especially in younger households, suggesting different impacts of housing price increases on rental expectations.
In Portugal, improved accessibility to housing markets contrasts with the community averages, despite a tendency towards higher effort rates, as indicated by Alves and Pinheiro (2017). They credit the job market's recovery for bolstering household incomes, thereby invigorating the real estate market. McGreal and Taltavull de La Paz (2013) emphasized income, accessibility, and structural features as pivotal to housing prices.
Fuller et al. (2020) discuss the relationship between housing prices, wealth indices, and financial asset price changes in Western Europe, predicting a significant impact on future generations. According to INE, household mobility demands are shaped by changes in family size and demographic shifts, with Varão (2019) attributing house price variability to demographic factors such as immigration and birth rates. Archer et al. (2010) found family size to be a dominant factor affecting housing prices in the Chicago market.
Johnes and Hyclak (1999) related average industry wages, unemployment rates, workforce strength, and average house prices in urban areas, finding interplays between local housing and labor markets, with unemployment and workforce strength affecting house prices in specific U.S. regions.
These insights collectively formulate research hypotheses investigating the relationship between population density, purchasing power, family size, and unemployment with housing values, providing a multifaceted view of the housing market's undercurrents.



This review synthesizes findings from multiple studies on the impact of external factors on real estate values. Xiao et al. (2020) employed a hedonic pricing model and spatial econometric model, finding that noisy public square dancing in Hangzhou, China, depreciates nearby housing prices by 5.8% to 13%. In contrast, Zambrano-Monserrate's (2016) study in Machala, Ecuador, emphasized the significance of public services like water supply and waste collection for rental apartment markets. In Madrid, Chasco Yrigoyen and Sánchez Reyes (2012) observed that air pollution negatively affects housing prices across all quantiles, with noise pollution only impacting luxury homes' prices. Fitch Osuna et al. (2013) explored real estate dynamics in San Nicolás de los Garza, Mexico, noting an increase in housing prices in noisier areas, suggesting demographic and location preferences outweigh other factors.
Ermisch (1996) highlighted the implications of aging demographics on housing demand in Britain, suggesting a substantial slowdown in growth rates due to population aging. Bayet et al. (1991) placed housing expenses as a primary concern for younger and older age groups in family budgets. Eichholtz and Lindenthal (2014) attributed high education, good health, and high income to sustained housing demand among aging populations in Europe. Conversely, Brasington et al. (2015) and Gyourko & Linneman (1996) found that higher educational attainment increases housing demand, suggesting a strong correlation between market values and educational levels.
These diverse studies collectively support the hypothesis that environmental quality, demographic shifts, and educational levels significantly influence housing values. This integration of global research presents a nuanced understanding of real estate economics, highlighting the complex interplay between urban externalities and market behaviors.
In his research conducted in China, Xu (2017) indicates that when the housing supply is at or near maximum capacity, there is increased pressure on supply, which can indeed drive up home prices. Katz and Rosen (1987) empirically demonstrated, using a large dataset on housing in the San Francisco Bay Area of California, that construction levies appear to have a significant impact on home prices. Their regression model's results suggest that home prices are between 17% and 38% higher in communities where growth moratoria and/or growth control plans are in place. The widespread use of such fees, in many communities, restricts the available supply response. The spread of these growth regulatory techniques to metropolitan areas outside California may have substantial negative effects on housing affordability (Katz and Rosen 1987). Consequently, the following research hypothesis can be considered:
H11: The number of available dwellings affects the value of housing.


% Head 2
\subsection{\uppercase{Data and Results}}
\subsubsection{Data}

\begin{itemize}
    \item Amostra
    \item Caracterização da Amostra - Tabelas descritivas 
\end{itemize}







\subsubsection{Factors in
the value of housing in Portugal}

\begin{itemize}
    \item Segundo modelo de regressão
\end{itemize}



% Sections that will go in second font
\subsection{\uppercase{Conclusions}}

The analysis presented in this paper provides a detailed exploration of the factors influencing housing prices across municipalities in Portugal, with a distinct emphasis on the period of transition from growth to stabilization in 2017. Through a refined multiple linear regression approach, we have delineated the impact of geographical, demographic, economic, and educational variables on property values within local contexts.
Our findings reveal a pronounced geographic determinant in housing valuation, asserting the premium placed on coastal and urban locations. The size of the average household emerges as a significant predictor, reflecting the market's response to family housing demands. Economic health, as measured by local unemployment rates, exhibits a predictable yet critical inverse correlation with housing prices, indicating the sensitivity of the real estate market to job availability. Moreover, the positive association between educational levels and housing prices emphasizes the value attributed to intellectual capital in residential desirability.
Notably, the study has highlighted the nuanced effect of homeownership rates on the market, contesting the straightforward view of ownership as a mere indicator of market heat. Instead, it is shown to interact with other factors in a way that could point to market maturity or saturation, suggesting a more complex scenario than previously understood.
These results carry implications for stakeholders in the housing market. Potential homebuyers can benefit from understanding how these variables interconnect, potentially guiding more informed decision-making. For policymakers, the insights offer concrete evidence to inform strategic planning, economic development initiatives, and educational investment to foster balanced growth and maintain market vitality.
The static nature of the data from a single year is a recognized limitation of this study, suggesting the need for ongoing research. As new census data becomes available, it will be imperative to re-evaluate these findings and observe how the interplay of these factors evolves over time. This continued analysis will be vital for adapting policies to the shifting realities of Portugal's diverse municipalities and ensuring that interventions remain relevant and effective in stabilizing and stimulating the housing market.


% Acknowledgement
\section{ACKNOWLEDGMENTS}
This work has been supported by national funds through FCT - Fundação para a Ciência e Tecnologia through project UIDB/04728/2020.

% References

\nocite{*}
\bibliographystyle{aipnum-cp}%
\bibliography{sample.bib}%


\end{document}
